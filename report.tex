% Options for packages loaded elsewhere
\PassOptionsToPackage{unicode}{hyperref}
\PassOptionsToPackage{hyphens}{url}
\PassOptionsToPackage{dvipsnames,svgnames,x11names}{xcolor}
%
\documentclass[
  letterpaper,
  DIV=11,
  numbers=noendperiod]{scrartcl}

\usepackage{amsmath,amssymb}
\usepackage{iftex}
\ifPDFTeX
  \usepackage[T1]{fontenc}
  \usepackage[utf8]{inputenc}
  \usepackage{textcomp} % provide euro and other symbols
\else % if luatex or xetex
  \usepackage{unicode-math}
  \defaultfontfeatures{Scale=MatchLowercase}
  \defaultfontfeatures[\rmfamily]{Ligatures=TeX,Scale=1}
\fi
\usepackage{lmodern}
\ifPDFTeX\else  
    % xetex/luatex font selection
\fi
% Use upquote if available, for straight quotes in verbatim environments
\IfFileExists{upquote.sty}{\usepackage{upquote}}{}
\IfFileExists{microtype.sty}{% use microtype if available
  \usepackage[]{microtype}
  \UseMicrotypeSet[protrusion]{basicmath} % disable protrusion for tt fonts
}{}
\makeatletter
\@ifundefined{KOMAClassName}{% if non-KOMA class
  \IfFileExists{parskip.sty}{%
    \usepackage{parskip}
  }{% else
    \setlength{\parindent}{0pt}
    \setlength{\parskip}{6pt plus 2pt minus 1pt}}
}{% if KOMA class
  \KOMAoptions{parskip=half}}
\makeatother
\usepackage{xcolor}
\setlength{\emergencystretch}{3em} % prevent overfull lines
\setcounter{secnumdepth}{-\maxdimen} % remove section numbering
% Make \paragraph and \subparagraph free-standing
\ifx\paragraph\undefined\else
  \let\oldparagraph\paragraph
  \renewcommand{\paragraph}[1]{\oldparagraph{#1}\mbox{}}
\fi
\ifx\subparagraph\undefined\else
  \let\oldsubparagraph\subparagraph
  \renewcommand{\subparagraph}[1]{\oldsubparagraph{#1}\mbox{}}
\fi

\usepackage{color}
\usepackage{fancyvrb}
\newcommand{\VerbBar}{|}
\newcommand{\VERB}{\Verb[commandchars=\\\{\}]}
\DefineVerbatimEnvironment{Highlighting}{Verbatim}{commandchars=\\\{\}}
% Add ',fontsize=\small' for more characters per line
\usepackage{framed}
\definecolor{shadecolor}{RGB}{241,243,245}
\newenvironment{Shaded}{\begin{snugshade}}{\end{snugshade}}
\newcommand{\AlertTok}[1]{\textcolor[rgb]{0.68,0.00,0.00}{#1}}
\newcommand{\AnnotationTok}[1]{\textcolor[rgb]{0.37,0.37,0.37}{#1}}
\newcommand{\AttributeTok}[1]{\textcolor[rgb]{0.40,0.45,0.13}{#1}}
\newcommand{\BaseNTok}[1]{\textcolor[rgb]{0.68,0.00,0.00}{#1}}
\newcommand{\BuiltInTok}[1]{\textcolor[rgb]{0.00,0.23,0.31}{#1}}
\newcommand{\CharTok}[1]{\textcolor[rgb]{0.13,0.47,0.30}{#1}}
\newcommand{\CommentTok}[1]{\textcolor[rgb]{0.37,0.37,0.37}{#1}}
\newcommand{\CommentVarTok}[1]{\textcolor[rgb]{0.37,0.37,0.37}{\textit{#1}}}
\newcommand{\ConstantTok}[1]{\textcolor[rgb]{0.56,0.35,0.01}{#1}}
\newcommand{\ControlFlowTok}[1]{\textcolor[rgb]{0.00,0.23,0.31}{#1}}
\newcommand{\DataTypeTok}[1]{\textcolor[rgb]{0.68,0.00,0.00}{#1}}
\newcommand{\DecValTok}[1]{\textcolor[rgb]{0.68,0.00,0.00}{#1}}
\newcommand{\DocumentationTok}[1]{\textcolor[rgb]{0.37,0.37,0.37}{\textit{#1}}}
\newcommand{\ErrorTok}[1]{\textcolor[rgb]{0.68,0.00,0.00}{#1}}
\newcommand{\ExtensionTok}[1]{\textcolor[rgb]{0.00,0.23,0.31}{#1}}
\newcommand{\FloatTok}[1]{\textcolor[rgb]{0.68,0.00,0.00}{#1}}
\newcommand{\FunctionTok}[1]{\textcolor[rgb]{0.28,0.35,0.67}{#1}}
\newcommand{\ImportTok}[1]{\textcolor[rgb]{0.00,0.46,0.62}{#1}}
\newcommand{\InformationTok}[1]{\textcolor[rgb]{0.37,0.37,0.37}{#1}}
\newcommand{\KeywordTok}[1]{\textcolor[rgb]{0.00,0.23,0.31}{#1}}
\newcommand{\NormalTok}[1]{\textcolor[rgb]{0.00,0.23,0.31}{#1}}
\newcommand{\OperatorTok}[1]{\textcolor[rgb]{0.37,0.37,0.37}{#1}}
\newcommand{\OtherTok}[1]{\textcolor[rgb]{0.00,0.23,0.31}{#1}}
\newcommand{\PreprocessorTok}[1]{\textcolor[rgb]{0.68,0.00,0.00}{#1}}
\newcommand{\RegionMarkerTok}[1]{\textcolor[rgb]{0.00,0.23,0.31}{#1}}
\newcommand{\SpecialCharTok}[1]{\textcolor[rgb]{0.37,0.37,0.37}{#1}}
\newcommand{\SpecialStringTok}[1]{\textcolor[rgb]{0.13,0.47,0.30}{#1}}
\newcommand{\StringTok}[1]{\textcolor[rgb]{0.13,0.47,0.30}{#1}}
\newcommand{\VariableTok}[1]{\textcolor[rgb]{0.07,0.07,0.07}{#1}}
\newcommand{\VerbatimStringTok}[1]{\textcolor[rgb]{0.13,0.47,0.30}{#1}}
\newcommand{\WarningTok}[1]{\textcolor[rgb]{0.37,0.37,0.37}{\textit{#1}}}

\providecommand{\tightlist}{%
  \setlength{\itemsep}{0pt}\setlength{\parskip}{0pt}}\usepackage{longtable,booktabs,array}
\usepackage{calc} % for calculating minipage widths
% Correct order of tables after \paragraph or \subparagraph
\usepackage{etoolbox}
\makeatletter
\patchcmd\longtable{\par}{\if@noskipsec\mbox{}\fi\par}{}{}
\makeatother
% Allow footnotes in longtable head/foot
\IfFileExists{footnotehyper.sty}{\usepackage{footnotehyper}}{\usepackage{footnote}}
\makesavenoteenv{longtable}
\usepackage{graphicx}
\makeatletter
\def\maxwidth{\ifdim\Gin@nat@width>\linewidth\linewidth\else\Gin@nat@width\fi}
\def\maxheight{\ifdim\Gin@nat@height>\textheight\textheight\else\Gin@nat@height\fi}
\makeatother
% Scale images if necessary, so that they will not overflow the page
% margins by default, and it is still possible to overwrite the defaults
% using explicit options in \includegraphics[width, height, ...]{}
\setkeys{Gin}{width=\maxwidth,height=\maxheight,keepaspectratio}
% Set default figure placement to htbp
\makeatletter
\def\fps@figure{htbp}
\makeatother

\KOMAoption{captions}{tableheading}
\makeatletter
\makeatother
\makeatletter
\makeatother
\makeatletter
\@ifpackageloaded{caption}{}{\usepackage{caption}}
\AtBeginDocument{%
\ifdefined\contentsname
  \renewcommand*\contentsname{Table of contents}
\else
  \newcommand\contentsname{Table of contents}
\fi
\ifdefined\listfigurename
  \renewcommand*\listfigurename{List of Figures}
\else
  \newcommand\listfigurename{List of Figures}
\fi
\ifdefined\listtablename
  \renewcommand*\listtablename{List of Tables}
\else
  \newcommand\listtablename{List of Tables}
\fi
\ifdefined\figurename
  \renewcommand*\figurename{Figure}
\else
  \newcommand\figurename{Figure}
\fi
\ifdefined\tablename
  \renewcommand*\tablename{Table}
\else
  \newcommand\tablename{Table}
\fi
}
\@ifpackageloaded{float}{}{\usepackage{float}}
\floatstyle{ruled}
\@ifundefined{c@chapter}{\newfloat{codelisting}{h}{lop}}{\newfloat{codelisting}{h}{lop}[chapter]}
\floatname{codelisting}{Listing}
\newcommand*\listoflistings{\listof{codelisting}{List of Listings}}
\makeatother
\makeatletter
\@ifpackageloaded{caption}{}{\usepackage{caption}}
\@ifpackageloaded{subcaption}{}{\usepackage{subcaption}}
\makeatother
\makeatletter
\@ifpackageloaded{tcolorbox}{}{\usepackage[skins,breakable]{tcolorbox}}
\makeatother
\makeatletter
\@ifundefined{shadecolor}{\definecolor{shadecolor}{rgb}{.97, .97, .97}}
\makeatother
\makeatletter
\makeatother
\makeatletter
\makeatother
\ifLuaTeX
  \usepackage{selnolig}  % disable illegal ligatures
\fi
\IfFileExists{bookmark.sty}{\usepackage{bookmark}}{\usepackage{hyperref}}
\IfFileExists{xurl.sty}{\usepackage{xurl}}{} % add URL line breaks if available
\urlstyle{same} % disable monospaced font for URLs
\hypersetup{
  pdftitle={Unraveling Wage Disparities: Gender and Other Determinants of Compensation in the Workplace},
  colorlinks=true,
  linkcolor={blue},
  filecolor={Maroon},
  citecolor={Blue},
  urlcolor={Blue},
  pdfcreator={LaTeX via pandoc}}

\title{Unraveling Wage Disparities: Gender and Other Determinants of
Compensation in the Workplace}
\usepackage{etoolbox}
\makeatletter
\providecommand{\subtitle}[1]{% add subtitle to \maketitle
  \apptocmd{\@title}{\par {\large #1 \par}}{}{}
}
\makeatother
\subtitle{Authors: Federico Di Gennaro, Elsa Farinella, Marco Scialanga}
\author{}
\date{}

\begin{document}
\maketitle
\ifdefined\Shaded\renewenvironment{Shaded}{\begin{tcolorbox}[sharp corners, breakable, interior hidden, frame hidden, borderline west={3pt}{0pt}{shadecolor}, enhanced, boxrule=0pt]}{\end{tcolorbox}}\fi

\hypertarget{data-and-problem-description}{%
\subsection{Data and Problem
Description}\label{data-and-problem-description}}

In an era of increasing awareness and emphasis on equal opportunities in
the workplace, the issue of wage discrimination continues to be a
subject of paramount concern. What are the factors that unfairly affect
workers' compensation? With the goal of addressing this issue, we dive
into a subset of the \(\textit{1985 Current Population Survey}\),
containing information about personal details and professional
backgrounds of 534 workers in the United States.

We aim to investigate potential correlations between hourly wages and
specific worker attributes. Specifically, our analysis will primarily
address potential gender-related wage disparities, while also accounting
for variables such as educational attainment, work experience, and
marital status.

\begin{Shaded}
\begin{Highlighting}[]
\CommentTok{\# Load data, select columns we are interested in}
\NormalTok{df }\OtherTok{\textless{}{-}} \FunctionTok{read.csv}\NormalTok{(}\StringTok{"data.csv"}\NormalTok{, }\AttributeTok{header =} \ConstantTok{TRUE}\NormalTok{, }\AttributeTok{sep =} \StringTok{","}\NormalTok{)}
\NormalTok{df }\OtherTok{\textless{}{-}}\NormalTok{ df[}\FunctionTok{c}\NormalTok{(}\StringTok{"WAGE"}\NormalTok{, }\StringTok{"OCCUPATION"}\NormalTok{, }\StringTok{"EDUCATION"}\NormalTok{, }
           \StringTok{"EXPERIENCE"}\NormalTok{, }\StringTok{"AGE"}\NormalTok{, }\StringTok{"SEX"}\NormalTok{, }\StringTok{"MARR"}\NormalTok{, }\StringTok{"RACE"}\NormalTok{)]}
\end{Highlighting}
\end{Shaded}

\begin{Shaded}
\begin{Highlighting}[]
\CommentTok{\# Transform categorical variables to factors for easier visualization}
\NormalTok{df}\SpecialCharTok{$}\NormalTok{SEX }\OtherTok{\textless{}{-}} \FunctionTok{as.factor}\NormalTok{(df}\SpecialCharTok{$}\NormalTok{SEX)}
\NormalTok{df}\SpecialCharTok{$}\NormalTok{MARR }\OtherTok{\textless{}{-}} \FunctionTok{as.factor}\NormalTok{(df}\SpecialCharTok{$}\NormalTok{MARR)}
\NormalTok{df}\SpecialCharTok{$}\NormalTok{RACE }\OtherTok{\textless{}{-}} \FunctionTok{as.factor}\NormalTok{(df}\SpecialCharTok{$}\NormalTok{RACE)}
\NormalTok{df}\SpecialCharTok{$}\NormalTok{OCCUPATION }\OtherTok{\textless{}{-}} \FunctionTok{as.factor}\NormalTok{(df}\SpecialCharTok{$}\NormalTok{OCCUPATION)}

\CommentTok{\# Check presence of missing values}
\FunctionTok{print}\NormalTok{(}\FunctionTok{c}\NormalTok{(}\StringTok{\textquotesingle{}Is there any missing value? \textquotesingle{}}\NormalTok{, }\FunctionTok{any}\NormalTok{(}\FunctionTok{is.na}\NormalTok{(df))))}
\end{Highlighting}
\end{Shaded}

\begin{verbatim}
[1] "Is there any missing value? " "FALSE"                       
\end{verbatim}

The variables we are interested in are:

\begin{itemize}
\item
  \texttt{WAGE}: hourly wage in dollars (numerical);
\item
  \texttt{OCCUPATION}: worker's occupation category (categorical -
  1=Manager, 2=Salespeople, 3=Office workers, 4=Manual workers,
  5=Professionals, 6=Other);
\item
  \texttt{EDUCATION}: level of education measured in years of
  schooling/university education (numerical);
\item
  \texttt{EXPERIENCE}: level of professional experience measured in
  years of full-time employment (numerical);
\item
  \texttt{AGE}: worker's age measured in years (numerical);
\item
  \texttt{SEX}: worker's gender (binary - 1 if female, 0 if male);
\item
  \texttt{MARR}: worker's marital status (binary - 1 if married or has a
  steady partner, 0 otherwise);
\item
  \texttt{RACE}: worker's ethnicity (categorical - 3 if Caucasian, 2 if
  Hispanic, 1 otherwise).
\end{itemize}

To get an idea of how our data is distributed, we take a look at each
variable individually using histograms, barplots, and boxplots.

\begin{Shaded}
\begin{Highlighting}[]
\CommentTok{\# Divide data in numerical and categorical for better visualization}
\NormalTok{numerical\_data }\OtherTok{\textless{}{-}}\NormalTok{ df[, }\FunctionTok{sapply}\NormalTok{(df, is.numeric)]}
\NormalTok{factor\_data }\OtherTok{\textless{}{-}}\NormalTok{ df[, }\FunctionTok{sapply}\NormalTok{(df, is.factor)]}
\end{Highlighting}
\end{Shaded}

\begin{Shaded}
\begin{Highlighting}[]
\CommentTok{\# Calculate mean for each variable}
\NormalTok{stats\_data }\OtherTok{\textless{}{-}}\NormalTok{ numerical\_data }\SpecialCharTok{\%\textgreater{}\%}
  \FunctionTok{pivot\_longer}\NormalTok{(}\FunctionTok{everything}\NormalTok{()) }\SpecialCharTok{\%\textgreater{}\%}
  \FunctionTok{group\_by}\NormalTok{(name) }\SpecialCharTok{\%\textgreater{}\%}
  \FunctionTok{summarize}\NormalTok{(}
    \AttributeTok{mean\_value =} \FunctionTok{mean}\NormalTok{(value, }\AttributeTok{na.rm =} \ConstantTok{TRUE}\NormalTok{),}
\NormalTok{  )}

\CommentTok{\# Histograms for numerical variables}
\NormalTok{numerical\_data }\SpecialCharTok{\%\textgreater{}\%}
  \FunctionTok{pivot\_longer}\NormalTok{(}\FunctionTok{everything}\NormalTok{()) }\SpecialCharTok{\%\textgreater{}\%}
  \FunctionTok{ggplot}\NormalTok{(}\FunctionTok{aes}\NormalTok{(value)) }\SpecialCharTok{+}
  \FunctionTok{geom\_histogram}\NormalTok{(}\AttributeTok{bins =} \DecValTok{30}\NormalTok{, }\AttributeTok{color=}\StringTok{"blue"}\NormalTok{) }\SpecialCharTok{+}
  \FunctionTok{geom\_vline}\NormalTok{(}\AttributeTok{data =}\NormalTok{ stats\_data, }\FunctionTok{aes}\NormalTok{(}\AttributeTok{xintercept =}\NormalTok{ mean\_value), }\AttributeTok{color =} \StringTok{"red"}\NormalTok{, }
             \AttributeTok{linetype =} \StringTok{"solid"}\NormalTok{, }\AttributeTok{linewidth =} \DecValTok{1}\NormalTok{) }\SpecialCharTok{+}
  \FunctionTok{facet\_wrap}\NormalTok{(}\SpecialCharTok{\textasciitilde{}}\NormalTok{ name, }\AttributeTok{scales =} \StringTok{"free"}\NormalTok{) }\SpecialCharTok{+}
  \FunctionTok{labs}\NormalTok{(}\AttributeTok{title =} \StringTok{"Histograms with Mean Lines for Numerical Variables"}\NormalTok{) }\SpecialCharTok{+}
  \FunctionTok{theme\_minimal}\NormalTok{() }\SpecialCharTok{+}
  \FunctionTok{theme}\NormalTok{(}\AttributeTok{plot.title =} \FunctionTok{element\_text}\NormalTok{(}\AttributeTok{size =} \DecValTok{14}\NormalTok{, }\AttributeTok{face =} \StringTok{"bold"}\NormalTok{, }\AttributeTok{hjust =} \FloatTok{0.5}\NormalTok{))}
\end{Highlighting}
\end{Shaded}

\begin{figure}[H]

{\centering \includegraphics{report_files/figure-pdf/unnamed-chunk-5-1.pdf}

}

\end{figure}

\begin{Shaded}
\begin{Highlighting}[]
\CommentTok{\# Barplots for categorical variables}
\NormalTok{factor\_data }\SpecialCharTok{\%\textgreater{}\%}
  \FunctionTok{pivot\_longer}\NormalTok{(}\FunctionTok{everything}\NormalTok{()) }\SpecialCharTok{\%\textgreater{}\%}
  \FunctionTok{ggplot}\NormalTok{(}\FunctionTok{aes}\NormalTok{(}\AttributeTok{x =}\NormalTok{ value)) }\SpecialCharTok{+}
  \FunctionTok{geom\_bar}\NormalTok{(}\AttributeTok{color=}\StringTok{"blue"}\NormalTok{) }\SpecialCharTok{+}
  \FunctionTok{facet\_wrap}\NormalTok{(}\SpecialCharTok{\textasciitilde{}}\NormalTok{ name, }\AttributeTok{scales =} \StringTok{"free"}\NormalTok{) }\SpecialCharTok{+}
  \FunctionTok{theme\_minimal}\NormalTok{() }\SpecialCharTok{+}
  \FunctionTok{theme}\NormalTok{(}\AttributeTok{plot.title =} \FunctionTok{element\_text}\NormalTok{(}\AttributeTok{size =} \DecValTok{14}\NormalTok{, }\AttributeTok{face =} \StringTok{"bold"}\NormalTok{, }\AttributeTok{hjust =} \FloatTok{0.5}\NormalTok{)) }\SpecialCharTok{+}
  \FunctionTok{labs}\NormalTok{(}\AttributeTok{title =} \StringTok{"Barplots for Categorical Variables"}\NormalTok{)}
\end{Highlighting}
\end{Shaded}

\begin{figure}[H]

{\centering \includegraphics{report_files/figure-pdf/unnamed-chunk-6-1.pdf}

}

\end{figure}

From the histograms above, we can see that our outcome variable WAGE is
skewed right with a mean of around \$9 per hour. Below we will see if
and how this distribution varies for different subsets of our data
(e.g.~males / females, married / not married). From the histograms of
the variables AGE and EXPERIENCE, we note that we have a pretty good
representation of both young and experienced workers, with the majority
being 25-35 years old and having 5-20 years of work experience.
Furthermore, we can see that the most common education level in the data
is high school, which corresponds to 12 years of education, with some
university students as well.

The two genders are represented quite evenly. There are about twice as
many married individuals than not married. The majority of occupations
is in the ``other'' category, with the rest of the jobs represented
somewhat equally. On the other hand, the vast majority of the
individuals in the dataset is Caucasian, with much fewer representants
of other ethnicities. This is a potential weakness of our data: it would
be useful to have a more even representation of ethnicities for a better
analysis.

Below are the descriptive statistics of the numerical variables of our
dataset to get more specific information regarding each one.

\begin{Shaded}
\begin{Highlighting}[]
\CommentTok{\# Descriptive statistics of numerical variables}
\FunctionTok{summary}\NormalTok{(numerical\_data)}
\end{Highlighting}
\end{Shaded}

\begin{verbatim}
      WAGE          EDUCATION       EXPERIENCE         AGE       
 Min.   : 1.000   Min.   : 2.00   Min.   : 0.00   Min.   :18.00  
 1st Qu.: 5.250   1st Qu.:12.00   1st Qu.: 8.00   1st Qu.:28.00  
 Median : 7.780   Median :12.00   Median :15.00   Median :35.00  
 Mean   : 9.024   Mean   :13.02   Mean   :17.82   Mean   :36.83  
 3rd Qu.:11.250   3rd Qu.:15.00   3rd Qu.:26.00   3rd Qu.:44.00  
 Max.   :44.500   Max.   :18.00   Max.   :55.00   Max.   :64.00  
\end{verbatim}

\hypertarget{analysis-of-correlation-between-wage-and-other-variables}{%
\subsection{Analysis of Correlation between Wage and Other
Variables}\label{analysis-of-correlation-between-wage-and-other-variables}}

We now begin investigating how hourly wage is correlated with personal
details of each workers, such as their gender, ethnicity, education,
etc.

\begin{Shaded}
\begin{Highlighting}[]
\CommentTok{\# New categorical feature for level of education}
\NormalTok{df}\SpecialCharTok{$}\NormalTok{level\_edu }\OtherTok{\textless{}{-}} \FunctionTok{ifelse}\NormalTok{(df}\SpecialCharTok{$}\NormalTok{EDUCATION}\SpecialCharTok{\textless{}=}\DecValTok{12}\NormalTok{, }\StringTok{"High School"}\NormalTok{, }
                \FunctionTok{ifelse}\NormalTok{(df}\SpecialCharTok{$}\NormalTok{EDUCATION}\SpecialCharTok{\textgreater{}}\DecValTok{12} \SpecialCharTok{\&}\NormalTok{ df}\SpecialCharTok{$}\NormalTok{EDUCATION}\SpecialCharTok{\textless{}=}\DecValTok{15}\NormalTok{, }\StringTok{"Bachelor"}\NormalTok{, }
                       \StringTok{"Master"}\NormalTok{))}

\CommentTok{\# Boxplots}
\FunctionTok{par}\NormalTok{(}\AttributeTok{mfrow=}\FunctionTok{c}\NormalTok{(}\DecValTok{2}\NormalTok{,}\DecValTok{2}\NormalTok{)) }\CommentTok{\# better visualizations}
\FunctionTok{boxplot}\NormalTok{(df}\SpecialCharTok{$}\NormalTok{WAGE }\SpecialCharTok{\textasciitilde{}}\NormalTok{ df}\SpecialCharTok{$}\NormalTok{level\_edu, }\AttributeTok{col=}\StringTok{"white"}\NormalTok{, }\AttributeTok{ylim=}\FunctionTok{c}\NormalTok{(}\DecValTok{0}\NormalTok{,}\DecValTok{30}\NormalTok{), }
        \AttributeTok{xlab=}\StringTok{"Levels of Education"}\NormalTok{, }\AttributeTok{ylab=}\StringTok{"Hourly wage"}\NormalTok{, }
        \AttributeTok{main=}\StringTok{"Hourly Wage Distribution by }\SpecialCharTok{\textbackslash{}n}\StringTok{ Levels of Education"}\NormalTok{)}

\FunctionTok{boxplot}\NormalTok{(df}\SpecialCharTok{$}\NormalTok{WAGE }\SpecialCharTok{\textasciitilde{}}\NormalTok{ df}\SpecialCharTok{$}\NormalTok{RACE, }\AttributeTok{col=}\StringTok{"white"}\NormalTok{, }\AttributeTok{ylim=}\FunctionTok{c}\NormalTok{(}\DecValTok{0}\NormalTok{,}\DecValTok{30}\NormalTok{),}
        \AttributeTok{names=}\FunctionTok{c}\NormalTok{(}\StringTok{"Other"}\NormalTok{, }\StringTok{"Hispanic"}\NormalTok{, }\StringTok{"Caucasian"}\NormalTok{), }\AttributeTok{xlab=}\StringTok{"Race"}\NormalTok{, }
        \AttributeTok{ylab=}\StringTok{"Hourly wage"}\NormalTok{, }\AttributeTok{main=}\StringTok{"Hourly Wage Distribution by Race"}\NormalTok{) }

\FunctionTok{boxplot}\NormalTok{(df}\SpecialCharTok{$}\NormalTok{WAGE }\SpecialCharTok{\textasciitilde{}}\NormalTok{ df}\SpecialCharTok{$}\NormalTok{MARR, }\AttributeTok{col=}\StringTok{"white"}\NormalTok{, }\AttributeTok{ylim=}\FunctionTok{c}\NormalTok{(}\DecValTok{0}\NormalTok{,}\DecValTok{30}\NormalTok{), }
        \AttributeTok{names=}\FunctionTok{c}\NormalTok{(}\StringTok{"No"}\NormalTok{, }\StringTok{"Yes"}\NormalTok{), }\AttributeTok{xlab=}\StringTok{"Married"}\NormalTok{, }\AttributeTok{ylab=}\StringTok{"Hourly wage"}\NormalTok{, }
        \AttributeTok{main=}\StringTok{"Hourly Wage Distribution by Marital Status"}\NormalTok{) }

\FunctionTok{boxplot}\NormalTok{(df}\SpecialCharTok{$}\NormalTok{WAGE }\SpecialCharTok{\textasciitilde{}}\NormalTok{ df}\SpecialCharTok{$}\NormalTok{SEX, }\AttributeTok{col=}\StringTok{"white"}\NormalTok{, }\AttributeTok{ylim=}\FunctionTok{c}\NormalTok{(}\DecValTok{0}\NormalTok{,}\DecValTok{30}\NormalTok{), }
        \AttributeTok{names=}\FunctionTok{c}\NormalTok{(}\StringTok{"Male"}\NormalTok{, }\StringTok{"Female"}\NormalTok{), }\AttributeTok{xlab=}\StringTok{"Married"}\NormalTok{, }\AttributeTok{ylab=}\StringTok{"Hourly wage"}\NormalTok{, }
        \AttributeTok{main=}\StringTok{"Hourly Wage Distribution by Gender"}\NormalTok{) }
\end{Highlighting}
\end{Shaded}

\begin{figure}[H]

{\centering \includegraphics{report_files/figure-pdf/unnamed-chunk-8-1.pdf}

}

\end{figure}

Through the four boxplots above, we can see how the distribution of the
WAGE variable differs depending on education level, ethnicity, marital
status, and gender.

To generate the first boxplot on the topleft, we built a new categorical
feature for better interpretability: ``high school'' (12 years or less
of education), ``bachelor'' (12 - 15 years), ``master'' (more than 15
years). The boxplot illustrating the distribution of hourly wages across
the three educational levels indicates that individuals with a Master's
degree typically have a marginally higher median wage than those with
either a Bachelor's degree or a high school education, as expected.

Furthermore, the boxplot representing the hourly wage distribution
across three racial categories reveals that Caucasians possess the
highest median wage. They are followed closely by the Hispanic group,
while the ``Other'' category registers the lowest median wage. A
possible explanation is that, in the United States, certain minorities
are often living in worse socioeconomic conditions with fewer job
opportunities and a lower access to high-level education.

Upon analyzing the wage distribution based on marital status, we note
that individuals who are either married or have a stable partner have a
higher median wage compared to their unmarried counterparts. A possible
explanation would be that older people generally earn more and people
who are married are, on average, older than people who are not. Further
analysis with more data would be needed to make final conclusions on
this topic.

The boxplot in which the wage distribution for the two different genders
is compared suggests that the median wage for males exceeds that of
females, highlighting a potential gender wage disparity. To further
investigate this behavior, below we plot the distribution of the
numerical variables and we get their correlation by disaggregating data
with respect to the sex of the individual.

\begin{Shaded}
\begin{Highlighting}[]
\CommentTok{\# Compute mean for graph}
\NormalTok{mean\_men }\OtherTok{\textless{}{-}} \FunctionTok{mean}\NormalTok{(df}\SpecialCharTok{$}\NormalTok{WAGE[df}\SpecialCharTok{$}\NormalTok{SEX }\SpecialCharTok{==} \StringTok{"0"}\NormalTok{], }\AttributeTok{na.rm =} \ConstantTok{TRUE}\NormalTok{)}
\NormalTok{mean\_women }\OtherTok{\textless{}{-}} \FunctionTok{mean}\NormalTok{(df}\SpecialCharTok{$}\NormalTok{WAGE[df}\SpecialCharTok{$}\NormalTok{SEX }\SpecialCharTok{==} \StringTok{"1"}\NormalTok{], }\AttributeTok{na.rm =} \ConstantTok{TRUE}\NormalTok{)}

\CommentTok{\# Density curves for men\textquotesingle{}s and women\textquotesingle{}s wages}
\FunctionTok{ggplot}\NormalTok{(df, }\FunctionTok{aes}\NormalTok{(}\AttributeTok{x=}\NormalTok{WAGE, }\AttributeTok{fill=}\NormalTok{SEX, }\AttributeTok{alpha=}\FloatTok{0.5}\NormalTok{)) }\SpecialCharTok{+} 
  \FunctionTok{geom\_density}\NormalTok{(}\FunctionTok{aes}\NormalTok{(}\AttributeTok{y=}\FunctionTok{after\_stat}\NormalTok{(scaled)), }\AttributeTok{position=}\StringTok{"identity"}\NormalTok{) }\SpecialCharTok{+}
  \FunctionTok{scale\_alpha}\NormalTok{(}\AttributeTok{guide=}\StringTok{\textquotesingle{}none\textquotesingle{}}\NormalTok{) }\SpecialCharTok{+}
  \FunctionTok{scale\_color\_manual}\NormalTok{(}\AttributeTok{values =} \FunctionTok{c}\NormalTok{(}\StringTok{"0"} \OtherTok{=} \StringTok{"blue"}\NormalTok{, }\StringTok{"1"} \OtherTok{=} \StringTok{"pink"}\NormalTok{),}
                     \AttributeTok{labels =} \FunctionTok{c}\NormalTok{(}\StringTok{"0"} \OtherTok{=} \StringTok{"Men"}\NormalTok{, }\StringTok{"1"} \OtherTok{=} \StringTok{"Women"}\NormalTok{)) }\SpecialCharTok{+}
  \FunctionTok{scale\_fill\_manual}\NormalTok{(}\AttributeTok{values =} \FunctionTok{c}\NormalTok{(}\StringTok{"0"} \OtherTok{=} \StringTok{"blue"}\NormalTok{, }\StringTok{"1"} \OtherTok{=} \StringTok{"pink"}\NormalTok{),}
                     \AttributeTok{labels =} \FunctionTok{c}\NormalTok{(}\StringTok{"0"} \OtherTok{=} \StringTok{"Men"}\NormalTok{, }\StringTok{"1"} \OtherTok{=} \StringTok{"Women"}\NormalTok{)) }\SpecialCharTok{+}
  \FunctionTok{theme\_minimal}\NormalTok{() }\SpecialCharTok{+}
  \FunctionTok{geom\_vline}\NormalTok{(}\FunctionTok{aes}\NormalTok{(}\AttributeTok{xintercept=}\NormalTok{mean\_men), }\AttributeTok{color=}\StringTok{"blue"}\NormalTok{, }\AttributeTok{linetype=}\StringTok{"solid"}\NormalTok{, }\AttributeTok{linewidth=}\DecValTok{1}\NormalTok{) }\SpecialCharTok{+} 
  \FunctionTok{geom\_vline}\NormalTok{(}\FunctionTok{aes}\NormalTok{(}\AttributeTok{xintercept=}\NormalTok{mean\_women), }\AttributeTok{color=}\StringTok{"pink"}\NormalTok{, }\AttributeTok{linetype=}\StringTok{"solid"}\NormalTok{, }\AttributeTok{linewidth=}\DecValTok{1}\NormalTok{) }\SpecialCharTok{+}
  \FunctionTok{labs}\NormalTok{(}\AttributeTok{y=}\StringTok{"Density"}\NormalTok{, }\AttributeTok{title =} \StringTok{"Density Curve of Wages for Men and Women with Mean Lines"}\NormalTok{) }\SpecialCharTok{+}
  \FunctionTok{theme}\NormalTok{(}
    \AttributeTok{axis.text.x =} \FunctionTok{element\_text}\NormalTok{(}\AttributeTok{size =} \DecValTok{14}\NormalTok{),}
    \AttributeTok{axis.text.y =} \FunctionTok{element\_text}\NormalTok{(}\AttributeTok{size =} \DecValTok{14}\NormalTok{),}
    \AttributeTok{legend.position =} \StringTok{"bottom"}\NormalTok{,}
    \AttributeTok{legend.text =} \FunctionTok{element\_text}\NormalTok{(}\AttributeTok{size =} \DecValTok{14}\NormalTok{),}
    \AttributeTok{legend.title =} \FunctionTok{element\_text}\NormalTok{(}\AttributeTok{size =} \DecValTok{16}\NormalTok{),}
    \AttributeTok{plot.title =} \FunctionTok{element\_text}\NormalTok{(}\AttributeTok{face =} \StringTok{"bold"}\NormalTok{, }\AttributeTok{hjust =} \FloatTok{0.5}\NormalTok{, }\AttributeTok{size =} \DecValTok{18}\NormalTok{))}
\end{Highlighting}
\end{Shaded}

\begin{figure}[H]

{\centering \includegraphics{report_files/figure-pdf/unnamed-chunk-9-1.pdf}

}

\end{figure}

\begin{itemize}
\tightlist
\item
  \textbf{Correlation of Wage with Age and Experience by Gender}
\end{itemize}

\begin{longtable}[]{@{}ccc@{}}
\toprule\noalign{}
& Age & Experience \\
\midrule\noalign{}
\endhead
\bottomrule\noalign{}
\endlastfoot
\textbf{Men} & 0.284 & 0.186 \\
\textbf{Women} & 0.092 & 0.003 \\
\end{longtable}

The difference between the two distributions above is clear. Although
the two density curves have the same shape (resembling a log normal
distribution), the men's curve is moved to the right of the women's, a
discrepancy that is well summarised by the distance between the two
means.

In addition, from the table above, we can see that variable WAGE for the
two genders behaves quite differently with respect to EXPERIENCE and
AGE: for men, wage is positively correlated with experience and age,
while for women the correlation is absent. This indicates that women's
wages, unlike mens', tend not to increase as female workers grow older
and become more experienced at their jobs.

In the analysis ahead, we will mainly focus on this disparity, looking
deeper into factors that might cause it.

\begin{Shaded}
\begin{Highlighting}[]
\CommentTok{\# Remove the outlier}
\NormalTok{df }\OtherTok{\textless{}{-}}\NormalTok{ df[df}\SpecialCharTok{$}\NormalTok{WAGE }\SpecialCharTok{\textless{}}\DecValTok{40}\NormalTok{, ]}
\NormalTok{wage\_women }\OtherTok{\textless{}{-}}\NormalTok{ df[df}\SpecialCharTok{$}\NormalTok{SEX}\SpecialCharTok{==}\DecValTok{1}\NormalTok{, ]  }
\NormalTok{wage\_men }\OtherTok{\textless{}{-}}\NormalTok{ df[df}\SpecialCharTok{$}\NormalTok{SEX}\SpecialCharTok{==}\DecValTok{0}\NormalTok{, ]}

\CommentTok{\# Calculate common axis limits}
\NormalTok{y\_limits }\OtherTok{\textless{}{-}} \FunctionTok{range}\NormalTok{(wage\_women}\SpecialCharTok{$}\NormalTok{WAGE, wage\_men}\SpecialCharTok{$}\NormalTok{WAGE)}
\NormalTok{x\_limits }\OtherTok{\textless{}{-}} \FunctionTok{range}\NormalTok{(}\DecValTok{0}\NormalTok{, }\FunctionTok{max}\NormalTok{(wage\_women}\SpecialCharTok{$}\NormalTok{EXPERIENCE, wage\_men}\SpecialCharTok{$}\NormalTok{EXPERIENCE))}

\CommentTok{\# Scatterplot with regression lines}
\NormalTok{plot1 }\OtherTok{\textless{}{-}} \FunctionTok{ggplot}\NormalTok{(}\AttributeTok{data =}\NormalTok{ wage\_women, }\FunctionTok{aes}\NormalTok{(}\AttributeTok{x =}\NormalTok{ EXPERIENCE, }\AttributeTok{y =}\NormalTok{ WAGE)) }\SpecialCharTok{+}
  \FunctionTok{geom\_point}\NormalTok{(}\AttributeTok{colour=}\StringTok{"pink"}\NormalTok{) }\SpecialCharTok{+}  
  \FunctionTok{geom\_smooth}\NormalTok{(}\AttributeTok{method=}\StringTok{"lm"}\NormalTok{, }\AttributeTok{se=}\NormalTok{T, }\AttributeTok{colour=}\StringTok{"red"}\NormalTok{) }\SpecialCharTok{+}  
  \FunctionTok{labs}\NormalTok{(}\AttributeTok{x=}\StringTok{"Work Experience"}\NormalTok{, }\AttributeTok{y=}\StringTok{"Hourly Wage"}\NormalTok{, }\AttributeTok{subtitle =} \StringTok{"Women"}\NormalTok{) }\SpecialCharTok{+}
  \FunctionTok{theme\_minimal}\NormalTok{() }\SpecialCharTok{+}
  \FunctionTok{theme}\NormalTok{(}\AttributeTok{panel.grid.major =} \FunctionTok{element\_blank}\NormalTok{(), }
        \AttributeTok{panel.border =} \FunctionTok{element\_rect}\NormalTok{(}\AttributeTok{fill =} \ConstantTok{NA}\NormalTok{, }\AttributeTok{color =} \StringTok{"black"}\NormalTok{), }
        \AttributeTok{plot.title =} \FunctionTok{element\_text}\NormalTok{(}\AttributeTok{face =} \StringTok{"bold"}\NormalTok{, }\AttributeTok{hjust =} \FloatTok{0.5}\NormalTok{)) }\SpecialCharTok{+}
  \FunctionTok{ylim}\NormalTok{(y\_limits) }\SpecialCharTok{+}
  \FunctionTok{xlim}\NormalTok{(x\_limits)}

\NormalTok{plot2 }\OtherTok{\textless{}{-}} \FunctionTok{ggplot}\NormalTok{(}\AttributeTok{data =}\NormalTok{ wage\_men, }\FunctionTok{aes}\NormalTok{(}\AttributeTok{x =}\NormalTok{ EXPERIENCE, }\AttributeTok{y =}\NormalTok{ WAGE)) }\SpecialCharTok{+}
  \FunctionTok{geom\_point}\NormalTok{(}\AttributeTok{colour=}\StringTok{"lightblue"}\NormalTok{) }\SpecialCharTok{+} 
  \FunctionTok{geom\_smooth}\NormalTok{(}\AttributeTok{method=}\StringTok{"lm"}\NormalTok{, }\AttributeTok{se=}\NormalTok{T, }\AttributeTok{colour=}\StringTok{"blue"}\NormalTok{) }\SpecialCharTok{+} 
  \FunctionTok{labs}\NormalTok{(}\AttributeTok{x=}\StringTok{"Work Experience"}\NormalTok{, }\AttributeTok{y=}\StringTok{"Hourly Wage"}\NormalTok{, }\AttributeTok{subtitle =} \StringTok{"Men"}\NormalTok{) }\SpecialCharTok{+}
  \FunctionTok{theme\_minimal}\NormalTok{() }\SpecialCharTok{+}
  \FunctionTok{theme}\NormalTok{(}\AttributeTok{panel.grid.major =} \FunctionTok{element\_blank}\NormalTok{(), }
        \AttributeTok{panel.border =} \FunctionTok{element\_rect}\NormalTok{(}\AttributeTok{fill =} \ConstantTok{NA}\NormalTok{, }\AttributeTok{color =} \StringTok{"black"}\NormalTok{),}
        \AttributeTok{plot.title =} \FunctionTok{element\_text}\NormalTok{(}\AttributeTok{face =} \StringTok{"bold"}\NormalTok{)) }\SpecialCharTok{+}
  \FunctionTok{ylim}\NormalTok{(y\_limits) }\SpecialCharTok{+}
  \FunctionTok{xlim}\NormalTok{(x\_limits)}

\FunctionTok{grid.arrange}\NormalTok{(plot1, plot2, }\AttributeTok{ncol =} \DecValTok{2}\NormalTok{, }
             \AttributeTok{top =} \FunctionTok{textGrob}\NormalTok{(}\StringTok{"Hourly Wage by Work Experience: }
\StringTok{                            }\SpecialCharTok{\textbackslash{}n}\StringTok{ A Comparison Between Men and Women"}\NormalTok{, }
                            \AttributeTok{gp =} \FunctionTok{gpar}\NormalTok{(}\AttributeTok{fontface =} \StringTok{"bold"}\NormalTok{, }\AttributeTok{fontsize =} \DecValTok{14}\NormalTok{)))}
\end{Highlighting}
\end{Shaded}

\begin{figure}[H]

{\centering \includegraphics{report_files/figure-pdf/unnamed-chunk-10-1.pdf}

}

\end{figure}

In the scatterplots above, the imposed regression lines with 95\%
confidence bands suggest that as work experience increases, the hourly
wage also tends to rise for both genders. However, the slope of the
trend line for men is much steeper, indicating a higher rate of wage
increase with experience compared to women. At the same levels of work
experience, women seem to have, on average, a lower hourly wage compared
to men, as indicated by the trend lines.

In summary, while both genders experience a rise in hourly wages with
increased work experience, there appears to be a gender wage gap with
men potentially earning more than women, especially as work experience
increases.

Next, we will look into wage disparities between the two genders, across
different age groups, to check for a similar trend as that found above.
First, we created a new column indicating the age group of each
individual, using a range of 3 years for young workers (to better detect
career evolutions that might be fairly quick at the beginning) and 4
years for older workers. Using this new feature, we generated the
following plot.

\begin{Shaded}
\begin{Highlighting}[]
\CommentTok{\# New categorical variable for better interpretability}
\NormalTok{df}\SpecialCharTok{$}\NormalTok{group\_age }\OtherTok{\textless{}{-}} \FunctionTok{ifelse}\NormalTok{(df}\SpecialCharTok{$}\NormalTok{AGE}\SpecialCharTok{\textless{}}\DecValTok{18}\NormalTok{, }\StringTok{"Under 18"}\NormalTok{, }
                \FunctionTok{ifelse}\NormalTok{(df}\SpecialCharTok{$}\NormalTok{AGE}\SpecialCharTok{\textgreater{}=}\DecValTok{18} \SpecialCharTok{\&}\NormalTok{ df}\SpecialCharTok{$}\NormalTok{AGE}\SpecialCharTok{\textless{}}\DecValTok{22}\NormalTok{, }\StringTok{"18{-}21"}\NormalTok{,}
                \FunctionTok{ifelse}\NormalTok{(df}\SpecialCharTok{$}\NormalTok{AGE}\SpecialCharTok{\textgreater{}=}\DecValTok{22} \SpecialCharTok{\&}\NormalTok{ df}\SpecialCharTok{$}\NormalTok{AGE}\SpecialCharTok{\textless{}}\DecValTok{26}\NormalTok{, }\StringTok{"22{-}25"}\NormalTok{,}
                \FunctionTok{ifelse}\NormalTok{(df}\SpecialCharTok{$}\NormalTok{AGE}\SpecialCharTok{\textgreater{}=}\DecValTok{26} \SpecialCharTok{\&}\NormalTok{ df}\SpecialCharTok{$}\NormalTok{AGE}\SpecialCharTok{\textless{}}\DecValTok{30}\NormalTok{, }\StringTok{"26{-}29"}\NormalTok{, }
                \FunctionTok{ifelse}\NormalTok{(df}\SpecialCharTok{$}\NormalTok{AGE}\SpecialCharTok{\textgreater{}=}\DecValTok{30} \SpecialCharTok{\&}\NormalTok{ df}\SpecialCharTok{$}\NormalTok{AGE}\SpecialCharTok{\textless{}}\DecValTok{35}\NormalTok{, }\StringTok{"30{-}34"}\NormalTok{,}
                \FunctionTok{ifelse}\NormalTok{(df}\SpecialCharTok{$}\NormalTok{AGE}\SpecialCharTok{\textgreater{}=}\DecValTok{35} \SpecialCharTok{\&}\NormalTok{ df}\SpecialCharTok{$}\NormalTok{AGE}\SpecialCharTok{\textless{}}\DecValTok{40}\NormalTok{, }\StringTok{"35{-}39"}\NormalTok{,}
                \FunctionTok{ifelse}\NormalTok{(df}\SpecialCharTok{$}\NormalTok{AGE}\SpecialCharTok{\textgreater{}=}\DecValTok{40} \SpecialCharTok{\&}\NormalTok{ df}\SpecialCharTok{$}\NormalTok{AGE}\SpecialCharTok{\textless{}}\DecValTok{45}\NormalTok{, }\StringTok{"40{-}44"}\NormalTok{,}
                \FunctionTok{ifelse}\NormalTok{(df}\SpecialCharTok{$}\NormalTok{AGE}\SpecialCharTok{\textgreater{}=}\DecValTok{45} \SpecialCharTok{\&}\NormalTok{ df}\SpecialCharTok{$}\NormalTok{AGE}\SpecialCharTok{\textless{}}\DecValTok{50}\NormalTok{, }\StringTok{"45{-}49"}\NormalTok{,}
                \FunctionTok{ifelse}\NormalTok{(df}\SpecialCharTok{$}\NormalTok{AGE}\SpecialCharTok{\textgreater{}=}\DecValTok{50} \SpecialCharTok{\&}\NormalTok{ df}\SpecialCharTok{$}\NormalTok{AGE}\SpecialCharTok{\textless{}}\DecValTok{55}\NormalTok{, }\StringTok{"50{-}54"}\NormalTok{,}
                \FunctionTok{ifelse}\NormalTok{(df}\SpecialCharTok{$}\NormalTok{AGE}\SpecialCharTok{\textgreater{}=}\DecValTok{55} \SpecialCharTok{\&}\NormalTok{ df}\SpecialCharTok{$}\NormalTok{AGE}\SpecialCharTok{\textless{}}\DecValTok{60}\NormalTok{, }\StringTok{"55{-}59"}\NormalTok{,}
                \FunctionTok{ifelse}\NormalTok{(df}\SpecialCharTok{$}\NormalTok{AGE}\SpecialCharTok{\textgreater{}=}\DecValTok{60} \SpecialCharTok{\&}\NormalTok{ df}\SpecialCharTok{$}\NormalTok{AGE}\SpecialCharTok{\textless{}}\DecValTok{65}\NormalTok{, }\StringTok{"60{-}64"}\NormalTok{,}\StringTok{" "}\NormalTok{)))))))))))}

\CommentTok{\# Build matrix for plot below}
\NormalTok{mat\_wage\_sex }\OtherTok{\textless{}{-}} \FunctionTok{aggregate}\NormalTok{(WAGE }\SpecialCharTok{\textasciitilde{}}\NormalTok{ SEX }\SpecialCharTok{+}\NormalTok{ group\_age, }\AttributeTok{data=}\NormalTok{df, }\AttributeTok{FUN=}\NormalTok{mean)}
\FunctionTok{colnames}\NormalTok{(mat\_wage\_sex) }\OtherTok{\textless{}{-}} \FunctionTok{c}\NormalTok{(}\StringTok{"gender"}\NormalTok{, }\StringTok{"group\_age"}\NormalTok{,}\StringTok{"avg\_salary"}\NormalTok{)}
\NormalTok{mat\_wage\_sex }\OtherTok{\textless{}{-}}\NormalTok{ mat\_wage\_sex }\SpecialCharTok{\%\textgreater{}\%} 
  \FunctionTok{spread}\NormalTok{(}\AttributeTok{key =}\NormalTok{ gender, }\AttributeTok{value =}\NormalTok{ avg\_salary)}

\NormalTok{matr\_wage\_sex }\OtherTok{\textless{}{-}}\NormalTok{ tidyr}\SpecialCharTok{::}\FunctionTok{gather}\NormalTok{(mat\_wage\_sex, }\AttributeTok{key =} \StringTok{"gender"}\NormalTok{, }
                               \AttributeTok{value =} \StringTok{"avg\_salary"}\NormalTok{, }\SpecialCharTok{{-}}\NormalTok{group\_age)}

\CommentTok{\# Lineplot}
\FunctionTok{ggplot}\NormalTok{(matr\_wage\_sex, }\FunctionTok{aes}\NormalTok{(}\AttributeTok{x =}\NormalTok{ group\_age, }\AttributeTok{y =}\NormalTok{ avg\_salary, }
                          \AttributeTok{color =}\NormalTok{ gender, }\AttributeTok{group =}\NormalTok{ gender)) }\SpecialCharTok{+}
  \FunctionTok{geom\_point}\NormalTok{(}\AttributeTok{size =} \DecValTok{3}\NormalTok{) }\SpecialCharTok{+} \FunctionTok{geom\_line}\NormalTok{() }\SpecialCharTok{+}
  \FunctionTok{scale\_color\_manual}\NormalTok{(}\AttributeTok{values =} \FunctionTok{c}\NormalTok{(}\StringTok{"0"} \OtherTok{=} \StringTok{"blue"}\NormalTok{, }\StringTok{"1"} \OtherTok{=} \StringTok{"pink"}\NormalTok{),}
                     \AttributeTok{labels =} \FunctionTok{c}\NormalTok{(}\StringTok{"0"} \OtherTok{=} \StringTok{"Men"}\NormalTok{, }\StringTok{"1"} \OtherTok{=} \StringTok{"Women"}\NormalTok{)) }\SpecialCharTok{+}  \CommentTok{\# Replace labels}
  \FunctionTok{labs}\NormalTok{(}\AttributeTok{title =} \StringTok{"Average Hourly Wage for Gender and Age Group"}\NormalTok{,}
       \AttributeTok{x =} \StringTok{"Age groups"}\NormalTok{, }\AttributeTok{y =} \StringTok{"Average hourly wage"}\NormalTok{) }\SpecialCharTok{+}
  \FunctionTok{theme\_minimal}\NormalTok{() }\SpecialCharTok{+}
  \FunctionTok{theme}\NormalTok{(}\AttributeTok{plot.title =} \FunctionTok{element\_text}\NormalTok{(}\AttributeTok{size =} \DecValTok{14}\NormalTok{, }\AttributeTok{face =} \StringTok{"bold"}\NormalTok{, }\AttributeTok{hjust =} \FloatTok{0.5}\NormalTok{))}
\end{Highlighting}
\end{Shaded}

\begin{figure}[H]

{\centering \includegraphics{report_files/figure-pdf/unnamed-chunk-11-1.pdf}

}

\end{figure}

The chart illustrates that, across all age groups examined, men's
average hourly wages consistently surpass those of women. In particular,
it emerges that in the subgroup of women the average hourly wage tends
to have a growth until the age of 30 while then it tends to stabilize
for the remaining groups considered. In contrast, for the male subgroup,
the upward trajectory in average hourly wages is evident until the age
of 45, post which it remains mainly stable.

Note that this could be due to unbalanced classes in terms of type of
occupation at different ages for men and women. For this reason, we
first investigate, for every age group we used in the previous plot, the
difference in type of occupation for men and women.

\begin{Shaded}
\begin{Highlighting}[]
\CommentTok{\# New categorical variable for better interpretability}
\NormalTok{df}\SpecialCharTok{$}\NormalTok{occupation\_class }\OtherTok{\textless{}{-}} \FunctionTok{ifelse}\NormalTok{(df}\SpecialCharTok{$}\NormalTok{OCCUPATION}\SpecialCharTok{==}\DecValTok{1} \SpecialCharTok{|}\NormalTok{ df}\SpecialCharTok{$}\NormalTok{OCCUPATION}\SpecialCharTok{==}\DecValTok{5}\NormalTok{,}
                              \StringTok{"High{-}compensation"}\NormalTok{, ifelse}
\NormalTok{                              (df}\SpecialCharTok{$}\NormalTok{OCCUPATION}\SpecialCharTok{==}\DecValTok{2} \SpecialCharTok{|}\NormalTok{ df}\SpecialCharTok{$}\NormalTok{OCCUPATION}\SpecialCharTok{==}\DecValTok{3} 
                              \SpecialCharTok{|}\NormalTok{df}\SpecialCharTok{$}\NormalTok{OCCUPATION}\SpecialCharTok{==}\DecValTok{4}\NormalTok{,}\StringTok{"Average{-}compensation"}\NormalTok{, }\StringTok{"Other"}\NormalTok{))}

\CommentTok{\# Create a custom facet variable}
\NormalTok{df}\SpecialCharTok{$}\NormalTok{facet\_label }\OtherTok{\textless{}{-}} \FunctionTok{ifelse}\NormalTok{(df}\SpecialCharTok{$}\NormalTok{SEX }\SpecialCharTok{==} \DecValTok{0}\NormalTok{, }\StringTok{"MEN"}\NormalTok{, }\StringTok{"WOMEN"}\NormalTok{)}

\CommentTok{\# Create separate grouped bar charts for men and women }
\CommentTok{\# stacked vertically with custom titles}
\FunctionTok{ggplot}\NormalTok{(df, }\FunctionTok{aes}\NormalTok{(}\AttributeTok{x =}\NormalTok{ group\_age, }\AttributeTok{fill =}\NormalTok{ occupation\_class)) }\SpecialCharTok{+}
  \FunctionTok{geom\_bar}\NormalTok{(}\AttributeTok{width =} \FloatTok{0.5}\NormalTok{, }\AttributeTok{position =} \FunctionTok{position\_dodge}\NormalTok{(}\AttributeTok{width =} \FloatTok{0.5}\NormalTok{)) }\SpecialCharTok{+}
  \FunctionTok{scale\_fill\_manual}\NormalTok{(}\AttributeTok{values =} \FunctionTok{c}\NormalTok{(}\StringTok{"blue"}\NormalTok{, }\StringTok{"red"}\NormalTok{, }\StringTok{"green"}\NormalTok{)) }\SpecialCharTok{+} 
  \FunctionTok{facet\_grid}\NormalTok{(facet\_label }\SpecialCharTok{\textasciitilde{}}\NormalTok{ .) }\SpecialCharTok{+}
  \FunctionTok{labs}\NormalTok{(}
    \AttributeTok{title =} \StringTok{"Distribution of Occupation by Age Group"}\NormalTok{,}
    \AttributeTok{x =} \StringTok{"Age groups"}\NormalTok{,}
    \AttributeTok{y =} \StringTok{"Count"}
\NormalTok{  ) }\SpecialCharTok{+}
  \FunctionTok{theme\_minimal}\NormalTok{() }\SpecialCharTok{+}
  \FunctionTok{theme}\NormalTok{(}
    \AttributeTok{plot.title =} \FunctionTok{element\_text}\NormalTok{(}\AttributeTok{size =} \DecValTok{18}\NormalTok{, }\AttributeTok{face =} \StringTok{"bold"}\NormalTok{, }\AttributeTok{hjust =} \FloatTok{0.5}\NormalTok{),}
    \AttributeTok{axis.title.x =} \FunctionTok{element\_text}\NormalTok{(}\AttributeTok{size =} \DecValTok{16}\NormalTok{),}
    \AttributeTok{axis.title.y =} \FunctionTok{element\_text}\NormalTok{(}\AttributeTok{size =} \DecValTok{16}\NormalTok{),}
    \AttributeTok{axis.text.x =} \FunctionTok{element\_text}\NormalTok{(}\AttributeTok{size =} \DecValTok{14}\NormalTok{),}
    \AttributeTok{axis.text.y =} \FunctionTok{element\_text}\NormalTok{(}\AttributeTok{size =} \DecValTok{14}\NormalTok{),}
    \AttributeTok{legend.position =} \StringTok{"bottom"}\NormalTok{,}
    \AttributeTok{legend.text =} \FunctionTok{element\_text}\NormalTok{(}\AttributeTok{size =} \DecValTok{14}\NormalTok{),}
    \AttributeTok{legend.title =} \FunctionTok{element\_text}\NormalTok{(}\AttributeTok{size =} \DecValTok{18}\NormalTok{),}
    \AttributeTok{strip.text =} \FunctionTok{element\_text}\NormalTok{(}\AttributeTok{size =} \DecValTok{14}\NormalTok{)}
\NormalTok{  )}
\end{Highlighting}
\end{Shaded}

\begin{figure}[H]

{\centering \includegraphics{report_files/figure-pdf/ggplot-1.pdf}

}

\end{figure}

First, we aggregated what we considered ``average-compensation'' jobs
(salespeople, office workers, manual workers) and what we considered
``high-compensation'' jobs (managers, professionals) to see whether men
and women have a difference in proportion within them or not.

From the plot above, we can see that the huge difference in wage for the
category 40-44 that we noticed (and to a lesser extent for the others as
well) can be associated with the different distribution of the
occupations for men and women. In particular, it is noticeable that the
majority of women in our data are employed in what we called
``average-compensation'' jobs, and the majority of men instead are doing
what we referred to as ``high-compensation'' jobs. For this reason, it
would be better to investigate the difference in salary within
categories of age by further dividing by occupation category, to
eliminate the confounding factor of different occupations between men
and women. Further, it is possible from the plot above to notice that a
lot of men in our data have a ``high-compensation'' job early in their
careers, which made us wonder: do men's careers advance faster than
women's careers? To answer such a multifaceted and complicated question,
more specific data and a deeper analysis would be needed.

\begin{Shaded}
\begin{Highlighting}[]
\CommentTok{\# Barplot}
\FunctionTok{barplot}\NormalTok{(}\FunctionTok{prop.table}\NormalTok{(}\FunctionTok{table}\NormalTok{(df}\SpecialCharTok{$}\NormalTok{SEX, df}\SpecialCharTok{$}\NormalTok{occupation\_class)), }\AttributeTok{beside=}\NormalTok{T, }
        \AttributeTok{col=}\FunctionTok{c}\NormalTok{(}\StringTok{"blue"}\NormalTok{, }\StringTok{"pink"}\NormalTok{), }\AttributeTok{horiz =}\NormalTok{ F,}
        \AttributeTok{ylim=}\FunctionTok{c}\NormalTok{(}\DecValTok{0}\NormalTok{,}\FloatTok{0.35}\NormalTok{), }\AttributeTok{main=}\StringTok{"Gender Differences in Occupational Sectors"}\NormalTok{, }
        \AttributeTok{ylab=}\StringTok{"Relative frequencies"}\NormalTok{)}
\FunctionTok{legend}\NormalTok{(}\AttributeTok{x=}\FloatTok{7.92}\NormalTok{, }\AttributeTok{y=}\FloatTok{0.35}\NormalTok{, }\AttributeTok{legend =} \FunctionTok{c}\NormalTok{(}\StringTok{"Male"}\NormalTok{, }\StringTok{"Female"}\NormalTok{), }
       \AttributeTok{col=}\FunctionTok{c}\NormalTok{(}\StringTok{"blue"}\NormalTok{, }\StringTok{"pink"}\NormalTok{), }\AttributeTok{pch=}\DecValTok{15}\NormalTok{)}
\FunctionTok{box}\NormalTok{()}
\end{Highlighting}
\end{Shaded}

\begin{figure}[H]

{\centering \includegraphics{report_files/figure-pdf/unnamed-chunk-12-1.pdf}

}

\end{figure}

To conclude, this final graph illustrates gender differences across
various occupational sectors. In ``average-compensation'' jobs, a higher
proportion of females are represented compared to males. In contrast,
``high-compensation'' jobs show a relatively even distribution between
the genders, with females slightly outnumbering males. For the ``Other''
category, males dominate significantly over their female counterparts.
On one hand, the higher proportion of women in ``average-compensation''
jobs might explain why they are paid less, in general, than men. On the
other hand, the fact that women and men are more or less equally
represented in ``high-compensation'' jobs might suggest that women are
compensated unfairly for this type of jobs compared to men.

\hypertarget{conclusion}{%
\subsection{Conclusion}\label{conclusion}}

The problem of wage disparity is very complex and affects a great amount
of workers around the world. In this report, we highlighted certain
trends related to this topic that we found in our data. First, we
noticed that there are evident differences in hourly wage when
considering different categories, such as sex, marital status and
education. Women are, on average, paid less than males, although this
could be explained by other factors. However, the absence of a
significant positive correlation between experience and hourly wage for
women highlights a serious problem. Furthermore, we found that married
and highly educated individuals tend to be better compensated than those
who aren't.

These or similar trends might very well still be present in today's
society, and an analysis of more modern data would be necessary to look
for similar patterns in the modern workplace. For this reason, it is
important to highlight that the results we have obtained should not be
regarded as definitive, but rather as a starting point for deeper
analysis.



\end{document}
